\documentclass[]{llncs}



\author{Marijn J.H. Heule}

\institute{Department of Software Technology\\
Delft University of Technology\\
\email{marijn@heule.nl}}


\title{Compact SAT Encoding of Spinpossible}

\begin{document}


\maketitle


\section{SAT translation}

$M$ refers to the number of moves.

\paragraph{State variables} The SAT encoding described in this paper . The meaning for variable $x_{i,j,k}$ is as follows: if $x_{i,j,k}$ is assigned to true, then square $i$ contains symbol $j$ at time $k$.
There are a special state variables $x_{i,0,k}$ to indicate whether the symbol on square $i$ is rotated. If $x_{i,0,k} = 1$ the symbol on square $i$ at time $k$ is upside down.  

\paragraph{Fixing initial and final state} Using the following unit clauses, the state variables of the final state can be fixed:

\begin{equation}
\bigwedge_{i=1}^9  (x_{i,i,M}) \land \bigwedge_{i=1}^9 \bigwedge_{j=0}^{i-1} (\bar x_{i,j ,M}) \land  \bigwedge_{i=1}^9 \bigwedge_{j=i+1}^{9} (\bar x_{i,j ,M})
\end{equation}

\paragraph{Equality constraints} The first type of auxiliary variables are {\em equality variables} $e_{i,k}$. The meaning of these variables is as follows: if $e_{i,k}$ is assigned to true, then then
tile $i$ is not modified during move $k$, or expressed in state variables: $\bar x_{i,j,k} \leftrightarrow \bar x_{i,j,k+1}$. The following clauses show the {\em equality constraints} that define
the equality variables:

\begin{equation}
\bigwedge_{i=1}^9 \bigwedge_{j=0}^9 \bigwedge_{k=0}^{M-1} \big( (\bar e_{i,k} \lor \bar x_{i,j,k} \lor x_{i,j,k+1}) \land (\bar e_{i,k} \lor x_{i,j,k} \lor \bar x_{i,j,k+1}) \big )
\end{equation}

\paragraph{Transition constraints} The second type of auxiliary variables are {\em transition variables} $t_{k,m}$.


\begin{equation}
\bigwedge_{k=0}^{M-1} (\bigvee_{m=1}^{36} t_{k,m}) \land \bigwedge_{i \notin S_m} (\bar t_{k,m} \lor e_{i,k}) \land \bigwedge_{i \in S_m} (\bar t_{k,m} \lor \bar e_{i,k})
\end{equation}



\paragraph{Symmetry breaking predicates}


\end{document}
